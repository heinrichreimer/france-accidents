\documentclass[usegeometry=true]{scrartcl}
\usepackage[english]{babel}
\usepackage[T1]{fontenc}
\usepackage{lmodern}
\usepackage[utf8]{inputenc}
\usepackage{hyperref}
\usepackage{amssymb}
\usepackage[left=2cm, right=2cm, top=2cm, bottom=2cm, bindingoffset=1cm, includeheadfoot]{geometry} % Don't change dimensions.
\usepackage[onehalfspacing]{setspace} % Don't change spacing.
\usepackage[backend=biber,style=numeric]{biblatex}
\usepackage{csquotes}
\usepackage{relsize}
\usepackage{xcolor}
\usepackage{xspace}
\usepackage{xurl}
\usepackage{listings}
\usepackage{listingsutf8}
\usepackage{booktabs}
\usepackage{tabularx}
\addbibresource{literature.bib}

% Listings
\definecolor{mediumgray}{gray}{0.60}
\definecolor{webiscodebasic}{rgb}{0.2,0.2,0.2}
\definecolor{webiscodekeyword}{rgb}{0.0,0.5,0.0}
\definecolor{webiscodekeywordself}{rgb}{0.7,0.4,0.6}
\definecolor{webiscodeidentifier}{rgb}{0.0,0.0,0.0}
\definecolor{webiscodecomment}{rgb}{0.25,0.5,0.5}
\definecolor{webiscodestring}{rgb}{0.75,0.12,0.12}
\definecolor{webiscodedecorator}{rgb}{0.6,0.3,0.0}
\lstdefinestyle{webisstyle}{
  basicstyle=\ttfamily\color{webiscodebasic},
  keywordstyle=\color{webiscodekeyword},
  identifierstyle=\color{webiscodeidentifier},
  commentstyle=\color{webiscodecomment},
  stringstyle=\color{webiscodestring},
  showstringspaces=false,
  frame=lines,
  framesep=0.7em,
  rulesep=0.5em,
  framerule=0.1em,
  keepspaces=true,
  tabsize=2,
  showtabs=false,
  numbers=none,
  literate=
    {->}{{\textrightarrow}}{2}
    {>=}{{\(\geq\)}}{2}
    {<=}{{\(\leq\)}}{2}
    {!=}{{\(\neq\)}}{2},
}
\lstloadlanguages{Haskell}
\lstset{
  style=webisstyle,
  language=Haskell,
  emph={[1]def,class},
  emphstyle={[1]\color{webiscodekeyword}\bfseries},
  emph={[2]self,cls},
  emphstyle={[2]\color{webiscodekeywordself}},
  emph={[3]@dataclass},
  emphstyle={[3]\color{webiscodedecorator}},
}

% Itemization:
\newcommand{\Ni}{(1)~}
\newcommand{\Nii}{(2)~}
\newcommand{\Niii}{(3)~}
\newcommand{\Niv}{(4)~}
\newcommand{\Nv}{(5)~}
\newcommand{\Na}{(a)~}
\newcommand{\Nb}{(b)~}
\newcommand{\Nc}{(c)~}
\newcommand{\Nd}{(d)~}
\newcommand{\Ne}{(e)~}

% Editing:
\newcommand{\todocite}{{\smaller\color{red}[CITE]}\xspace}
\newcommand{\todo}[1]{{\smaller\color{red}[#1]}}

\begin{document}

\subject{Project Report for the Module \\ \textquote{Information Retrieval und Visualisierung} \\ in Summer Semester~2022}
\title{Insights Towards Safer Roads}
\subtitle{Visualizing Accidents in France from~2005 to~2020}
\author{Jan Heinrich Reimer}
\date{\today}
\maketitle

\tableofcontents
\section{Introduction}
\label{introduction}
Despite decades of developing better technology and regulations to avoid road accidents, the global number of accidents is still high~\cite{Who2015}. In~2009, 1.5\,\% of EU citizens\footnote{\url{https://ec.europa.eu/eurostat/databrowser/bookmark/55b45b50-76ac-4cbf-9493-274087bd96fe}} reported injuries caused by road accidents. And even though in recent years the number of road deaths slightly declined in the European Union\footnote{\url{https://ec.europa.eu/eurostat/web/products-eurostat-news/-/DDN-20220511-1}}, in~2020 still 4 out of 100\,000 inhabitants died of road accidents. Among personal factors such as inexperiencedness or loss of control~\cite{RolisonRMF2018}, combusted cities~\cite{AlbalateF2021} and insufficient law enforcement~\cite{Who2015} are commonly cited as reasons behind those high numbers of accidents. Those are, however, factors that are not easy to change. Neither can urbanization be stopped nor can we circumvent all personal factors. This report instead seeks to develop new ideas towards road safety by visualizing French governmental open datasets of road accidents\footnote{\url{https://data.gouv.fr/en/datasets/bases-de-donnees-annuelles-des-accidents-corporels-de-la-circulation-routiere-annees-de-2005-a-2019/}} that happend in France in the period from~2005 to~2020. We presume that given France's representativity in Europe---with a population of 68~millions it is the second largest country within the 447~million population of the European Union~\footnote{as of January 1 2022, \url{https://ec.europa.eu/eurostat/databrowser/bookmark/40fd4e5f-3ace-4618-86a5-e9ecfea67ac2}}---some of our conclusions can also be applied to other European countries as well. In our Design Study~\cite{Munzner2008}, we propose new visualizations to explore three different dimensions that could cause road accidents: \Ni date- or time-related effects, \Nii the influence of the drivers and passengers, and \Niii prevalences of different types of accidents.

\subsection{Application Background}
\begin{table}
    \caption{Exemplary questions that should be answered by visualizions of French road accident dataset, with the typical background a person asking that question would likely have.}
    \label{table-questions}
    \begin{tabularx}{\linewidth}{Xp{0.3\linewidth}}
        \toprule
        \textbf{Question} & \textbf{Background} \\
        \midrule
        Are roads more dangerous in winter or summer? & citizen \\
        Do older or younger people drive more safely? & policy maker \\
        Are there geographical hotspots, e.g., in cities or rural areas? & citizen, policy maker \\
        If geographical hotspots occur, which other characteristics can explain the correlations. & policy maker \\
        Do the proportion of dead and injured persons correlate? & policy maker \\
        Where are unproportionately more people killed in traffic? & policy maker \\
        Do dedicated bicycle lanes make roads safer for cyclists? & citizen, policy maker, infrastructure planner \\
        Are wider roads safer than narrow roads? & citizen, infrastructure planner \\
        \bottomrule
    \end{tabularx}
\end{table}
With unacceptably high casualties in road traffic, our vizualization addresses the international initiative towards better road safety. The vizualizations should support people in avoiding particuarly dangerous regions, times, or other patterns. It should also provide officials with insights where additional policies or infrastructure might mitigate high numbers of traffic accidents. It is therefore important that broad trends can be expolored while at the same time giving the targetted users the option to dive into visualizations to gain insights on smaller details.
Specifically, our visualizations should answer the following exemplary questions listed in Table~\ref{table-questions}.

\subsection{Target Groups}
Our visualizations target citizens that commute on French roads as well as governments throughout the world. For the first group in particuar, we can not assume deep knowledge of public infrastructure or mechanical vocabulary~(e.g.,~they might not be able to intuitively distinguish between vehicles that weigh 1.5~or 3.5~tons). For policy makers on the other hand, we can assume such extended knowledge, assuming they were trained on special fields of public infrastucture. Both groups want to know where, when, and with whom accidents occur most often, that is, they want to explore patterns, citizens because they could then avoid such patterns and government officials because they could target such patterns by introducing laws or building new infrastructure~(e.g.,~a pedestrian bridge at a dangerous road crossing). 

\subsection{Overview and Contributions}
Our contributions are twofold: \Ni we make the French government's accident dataset more accessible by preprocessing, aggregating, and translating the data of 64~individual, French-named data files, and \Ni we develop an application with three interactive visualizations that help citizens as well as politicians in finding frequent dangers across French roads.
Working with highly multidimensional data, we apply patterns such as filtering and aggregation, partitioning into tree hierarchies~\cite{Shneiderman1992}, icon-based stick figure~\cite{PickettG1988} vizualization techniques, and interacive options to select or order specific dimensions of the dataset for each visualization.
We deploy our described visualizations as an interactive online demo\footnote{\url{https://heinrichreimer.github.io/france-accidents/}} and release the visualization code as well as code for preprocessing the open dataset on GitHub\footnote{\url{https://github.com/heinrichreimer/france-accidents}} under an open license.
The insights gianed from the visualizations introduced in this paper can be integrated in future policies and public infrastructure planning. This way roads can be made safer and people's lives can be saved.


\section{Data}
\todo{Beschreiben Sie vorhandenen Daten. Gehen Sie kritisch darauf ein, inwieweit sich die Daten für die Bearbeitung der Fragestellungen und dem Erreichen von Lösungen für die oben beschriebene Zielgruppen eignen. Haben sie die Daten sinnvoll mit weiteren Datenquellen ergänzt? Wenn ja, wie?
Erklären Sie die technische Bereitstellung der Daten.
Wie sind die Daten zugänglich? Welche Formate werden genutzt. Gibt es Besonderheiten beim Lesen der Formate?
Beschreiben Sie die Datenvorverarbeitung.
Welche Datenvorverarbeitungsschritte sind notwendig? Beschreiben Sie die einzelnen Schritte und begründen Sie sie, zum Beispiel warum werden manche Daten weggelassen, über welche Mengen werden Durchschnitte berechnet, warum sind die so berechneten Werte aussagekräftiger als andere Werte. Wenn möglich sollen sie die Datenvorverarbeitung in Elm programmieren, sodass ihre Anwendung auf eine Änderung der Rohdaten reagieren kan.}

\section{Visualizations}
\subsection{Analysis of Application Tasks}
\todo{Analysieren sie die konkreten Anwendungsaufgaben, die die Lösung des Zielproblems durch die Anwender:innen bearbeitet werden müssen. 
Welche sinnvollen mentalen Modelle helfen den Personen bei der Bearbeitung.
Welche Visualisierungen helfen den Personen, die die Software verwenden, sinnvolle mentale Modelle aufzubauen. 
Sind diese mentalen Modelle für sie notwendig, um die Aufgaben lösen zu können? Gehen Sie bei ihrer Argumentation von den Anwendungsaufgaben aus und kommen sie dann zu den mentalen Modellen, deren Aufbau durch Visualisierungen unterstützt wird.}
\subsection{Visualization Requirements}
\todo{Leiten sie Anforderungen an das Design der Visualisierungen ab, die sich durch ihre Analyse des Zielproblems ergeben.}
\subsection{Visualization Presentation}
\todo{Präsentieren sie die visuellen Abbildungen und Kodierungen der Daten und Interaktionsmöglichkeiten.
Sie müssen begründen, warum und wie gut ihre Designentscheidungen die erstellten Anforderungen erfüllen.
Weiterhin müssen sie begründen, warum die gewählte visuelle Kodierung der Daten für das zu lösenden Problem passend ist.
Typische Argumente würden hier auf Wahrnehmungsprinzipien und Theorie über Informationsvisualisierung verweisen. 
Die besten Begründungen diskutieren explizit die konkrete Auswahl der Visualisierungen im Kontext von mehreren verschiedenen Alternativen. 
Machen Sie hier nicht den Fehler, einfach nur Visualisierung aus den vorgegebenen Bereichen zu diskutieren, weil das in der Regel nicht sinnvoll ist.
Wenn sie sich für einen Scatterplot entschieden haben, ist ein Zeitreihendiagramm in der Regel keine Alternative.
Diskutieren Sie also nicht einfach Zeitreihendiagramme, weil sie in den Anforderungen an das Projekt neben Scatterplots stehen, sondern suchen sie nach echten alternativen Visualisierungen, die zum Aufbau eines vergleichbaren mentalen Modells führen.
Diskutieren Sie die Expressivität und die Effektivität der einzelnen Visualisierungen.}

\todo{Die eben beschriebenen Präsentationen und Begründungen sollen für jede der drei folgenden Visualisierungen durchgeführt werden. }
\subsubsection{First Visualization}
\subsubsection{Second Visualization}
We used a freely licensed equirectanguular map projection of france\footnote{\url{https://commons.wikimedia.org/wiki/File:France_location_map-Regions_and_departements-2016.svg}} as the background to the second visualization.
\subsubsection{Third Visualization}

\subsection{Interaction}
\todo{Die präsentierten Visualisierungstechniken müssen interaktiv zu einer Anwendung verknüpft werden.
Die Interaktion mit einer Visualisierung soll in den anderen Visualisierungen zu einer Änderung führen. 
Erklären Sie die möglichen Interaktionen mit den einzelnen Visualisierungen und die möglichen Verknüpfungen zwischen ihnen. Begründen Sie, warum die konkreten Interaktionen umgesetzt wurden und welche Zwecke für die Anwenderinnen mit ihnen unterstützt werden. Begründen Sie ebenfalls, warum sie andere Interaktionsmöglichkeiten nicht umgesetzt haben. Wenn sie keine der geforderten Interaktionen umsetzen, erhalten Sie im gesamten Projekt deutlichen Punktabzug.}

\section{Implementation}
\todo{Beschreiben Sie die Implementierung ihrer Visualisierungs-Anwendung in Elm. Stellen die Gliederung ihres Quellcodes vor. Haben Sie verschiedene Elm-Module erstellt. Was war aufwändig umzusetzen, was ließ sich mit dem vorhanden Code aus den Übungen relativ einfach umsetzen?}

\todo{Wie sieht die Elm-Datenstruktur für das Model aus, in dem die verschiedenen Zustände der Interaktion gespeichert werden können.}

\section{Use Cases}
\todo{Präsentieren sie für jede der drei Visualisierungen einen sinnvollen Anwendungsfall in dem ein bestimmter Fakt, ein Muster oder die Abwesenheit eines Musters visuell festgestellt wird. Begründen Sie, warum dieser Anwendungsfall wichtig für die Zielgruppe der Anwenderinnen ist. Diskutieren Sie weiterhin, ob die oben beschriebene Information auch mit anderen Visualisierungstechniken hätte gefunden werden können. Falls dies möglich wäre, vergleichen sie die den Aufwand und die Schwierigkeiten ihres Ansatzes und möglicher Alternativen.}
\subsection{First Visualization Use Case}
\subsection{Second Visualization Use Case}
\subsection{Third Visualization Use Case}

\section{Related Work}
\todo{Führen sie eine kurze Literatursuche in der wissenschaftlichen Literatur zu Informationsvisualisierung und Visual Analytics nach ähnlichen Anwendungen durch. Diskutieren Sie mindestens zwei Artikel. Stellen Sie Gemeinsamkeiten und Unterschiede dar.}

\section{Conclusion and Future Work}
\todo{Fassen sie die Beiträge ihre Visualisierungs-Anwendung zusammen. Wo bietet sie für die Personen der Zielgruppe einen echten Mehrwert?}

\todo{Was wären mögliche sinnvolle Erweiterungen, entweder auf der Ebene der Visualisierungen und/oder auf der Datenebene?}

\section*{Appendix: Git History}
% Generate with: git --no-pager log --all --graph --no-color --date=short --pretty='format:%h%d (%s, %ad)'
% \lstinputlisting[language=,basicstyle=\ttfamily\tiny]{git-log.txt}

\printbibliography

\end{document}

