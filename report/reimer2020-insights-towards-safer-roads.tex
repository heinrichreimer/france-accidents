\documentclass[usegeometry=true]{scrartcl}
\usepackage[english]{babel}
\usepackage[T1]{fontenc}
\usepackage{lmodern}
\usepackage[utf8]{inputenc}
\usepackage{hyperref}
\usepackage{amssymb}
\usepackage[left=2cm, right=2cm, top=2cm, bottom=2cm, bindingoffset=1cm, includeheadfoot]{geometry} % Don't change dimensions.
\usepackage[onehalfspacing]{setspace} % Don't change spacing.
\usepackage[backend=biber,style=numeric]{biblatex}
\usepackage{csquotes}
\usepackage{relsize}
\usepackage{xcolor}
\usepackage{xspace}
\usepackage{xurl}
\usepackage{listings}
\usepackage{listingsutf8}
\usepackage{booktabs}
\usepackage{tabularx}
\addbibresource{literature.bib}

% Listings
\definecolor{mediumgray}{gray}{0.60}
\definecolor{webiscodebasic}{rgb}{0.2,0.2,0.2}
\definecolor{webiscodekeyword}{rgb}{0.0,0.5,0.0}
\definecolor{webiscodekeywordself}{rgb}{0.7,0.4,0.6}
\definecolor{webiscodeidentifier}{rgb}{0.0,0.0,0.0}
\definecolor{webiscodecomment}{rgb}{0.25,0.5,0.5}
\definecolor{webiscodestring}{rgb}{0.75,0.12,0.12}
\definecolor{webiscodedecorator}{rgb}{0.6,0.3,0.0}
\lstdefinestyle{webisstyle}{
  basicstyle=\ttfamily\color{webiscodebasic},
  keywordstyle=\color{webiscodekeyword},
  identifierstyle=\color{webiscodeidentifier},
  commentstyle=\color{webiscodecomment},
  stringstyle=\color{webiscodestring},
  showstringspaces=false,
  frame=lines,
  framesep=0.7em,
  rulesep=0.5em,
  framerule=0.1em,
  keepspaces=true,
  tabsize=2,
  showtabs=false,
  numbers=none,
  literate=
    {->}{{\textrightarrow}}{2}
    {>=}{{\(\geq\)}}{2}
    {<=}{{\(\leq\)}}{2}
    {!=}{{\(\neq\)}}{2},
}
\lstloadlanguages{Haskell}
\lstset{
  style=webisstyle,
  language=Haskell,
  emph={[1]def,class},
  emphstyle={[1]\color{webiscodekeyword}\bfseries},
  emph={[2]self,cls},
  emphstyle={[2]\color{webiscodekeywordself}},
  emph={[3]@dataclass},
  emphstyle={[3]\color{webiscodedecorator}},
}

% Itemization:
\newcommand{\Ni}{(1)~}
\newcommand{\Nii}{(2)~}
\newcommand{\Niii}{(3)~}
\newcommand{\Niv}{(4)~}
\newcommand{\Nv}{(5)~}
\newcommand{\Na}{(a)~}
\newcommand{\Nb}{(b)~}
\newcommand{\Nc}{(c)~}
\newcommand{\Nd}{(d)~}
\newcommand{\Ne}{(e)~}

% Editing:
\newcommand{\todocite}{{\smaller\color{red}[CITE]}\xspace}
\newcommand{\todo}[1]{{\smaller\color{red}[#1]}}

\begin{document}

\subject{Project Report for the Module \\ \textquote{Information Retrieval und Visualisierung} \\ in Summer Semester~2022}
\title{Insights Towards Safer Roads}
\subtitle{Visualizing Accidents in France from~2005 to~2020}
\author{Jan Heinrich Reimer}
\date{\today}
\maketitle

\tableofcontents
\section{Introduction}
\label{introduction}
Despite decades of developing better technology and regulations to avoid road accidents, the global number of accidents is still high~\cite{Who2015}. In~2009, 1.5\,\% of EU citizens\footnote{\url{https://ec.europa.eu/eurostat/databrowser/bookmark/55b45b50-76ac-4cbf-9493-274087bd96fe}} reported injuries caused by road accidents. And even though in recent years the number of road deaths slightly declined in the European Union\footnote{\url{https://ec.europa.eu/eurostat/web/products-eurostat-news/-/DDN-20220511-1}}, in~2020 still 4 out of 100\,000 inhabitants died of road accidents. Among personal factors such as inexperiencedness or loss of control~\cite{RolisonRMF2018}, combusted cities~\cite{AlbalateF2021} and insufficient law enforcement~\cite{Who2015} are commonly cited as reasons behind those high numbers of accidents. Those are, however, factors that are not easy to change. Neither can urbanization be stopped nor can we circumvent all personal factors. This report instead seeks to develop new ideas towards road safety by visualizing French governmental open datasets of road accidents\footnote{\url{https://data.gouv.fr/en/datasets/bases-de-donnees-annuelles-des-accidents-corporels-de-la-circulation-routiere-annees-de-2005-a-2019/}} that happend in France in the period from~2005 to~2020. We presume that given France's representativity in Europe---with a population of 68~millions it is the second largest country within the 447~million population of the European Union~\footnote{as of January 1 2022, \url{https://ec.europa.eu/eurostat/databrowser/bookmark/40fd4e5f-3ace-4618-86a5-e9ecfea67ac2}}---some of our conclusions can also be applied to other European countries as well. In our Design Study~\cite{Munzner2008}, we propose new visualizations to explore three different dimensions that could cause road accidents: \Ni date- or time-related effects, \Nii the influence of the drivers and passengers, and \Niii prevalences of different types of accidents.

\subsection{Application Background}
\begin{table}
    \caption{Exemplary questions that should be answered by visualizions of French road accident dataset, with the typical background a person asking that question would likely have.}
    \label{table-questions}
    \begin{tabularx}{\linewidth}{Xp{0.3\linewidth}}
        \toprule
        \textbf{Question} & \textbf{Background} \\
        \midrule
        Are roads more dangerous in winter or summer? & citizen \\
        Do older or younger people drive more safely? & policy maker \\
        Are there geographical hotspots, e.g., in cities or rural areas? & citizen, policy maker \\
        If geographical hotspots occur, which other characteristics can explain the correlations. & policy maker \\
        Do the proportion of dead and injured persons correlate? & policy maker \\
        Where are unproportionately more people killed in traffic? & policy maker \\
        Do dedicated bicycle lanes make roads safer for cyclists? & citizen, policy maker, infrastructure planner \\
        Are wider roads safer than narrow roads? & citizen, infrastructure planner \\
        \bottomrule
    \end{tabularx}
\end{table}
With unacceptably high casualties in road traffic, our vizualization addresses the international initiative towards better road safety. The vizualizations should support people in avoiding particuarly dangerous regions, times, or other patterns. It should also provide officials with insights where additional policies or infrastructure might mitigate high numbers of traffic accidents. It is therefore important that broad trends can be expolored while at the same time giving the targetted users the option to dive into visualizations to gain insights on smaller details.
Specifically, our visualizations should answer the following exemplary questions listed in Table~\ref{table-questions}.

\subsection{Target Groups}
Our visualizations target citizens that commute on French roads as well as governments throughout the world. For the first group in particuar, we can not assume deep knowledge of public infrastructure or mechanical vocabulary~(e.g.,~they might not be able to intuitively distinguish between vehicles that weigh 1.5~or 3.5~tons). For policy makers on the other hand, we can assume such extended knowledge, assuming they were trained on special fields of public infrastucture. Both groups want to know where, when, and with whom accidents occur most often, that is, they want to explore patterns, citizens because they could then avoid such patterns and government officials because they could target such patterns by introducing laws or building new infrastructure~(e.g.,~a pedestrian bridge at a dangerous road crossing). 

\subsection{Overview and Contributions}
Our contributions are twofold: \Ni we make the French government's accident dataset more accessible by preprocessing, aggregating, and translating the data of 64~individual, French-named data files, and \Ni we develop an application with three interactive visualizations that help citizens as well as politicians in finding frequent dangers across French roads.
Working with highly multidimensional data, we apply patterns such as filtering and aggregation, partitioning into tree hierarchies~\cite{Shneiderman1992}, icon-based stick figure~\cite{PickettG1988} vizualization techniques, and interacive options to select or order specific dimensions of the dataset for each visualization.
We deploy our described visualizations as an interactive online demo\footnote{\url{https://heinrichreimer.github.io/france-accidents/}} and release the visualization code as well as code for preprocessing the open dataset on GitHub\footnote{\url{https://github.com/heinrichreimer/france-accidents}} under an open license.
The insights gianed from the visualizations introduced in this paper can be integrated in future policies and public infrastructure planning. This way roads can be made safer and people's lives can be saved.

\section{Data}
\label{data}
\begin{table}
    \caption{Files with accident data, released by the French government each year~(denoted by \textit{YYYY}), with description of its contents.}
    \label{table-files}
    \begin{tabularx}{\linewidth}{lX}
        \toprule
        \textbf{File Name} & \textbf{Contents} \\
        \midrule
        \textit{caracteristiques-YYYY.csv} & accident characteristics, time, place, etc. \\
        \textit{lieux-YYYY.csv} & accident location, road numbers, road characteristics, lanes etc. \\
        \textit{usagers-YYYY.csv} & persons involved, sex, birth, equipment, etc. \\
        \textit{vehicules-YYYY.csv} & vehicle type, maneuver, hit obstacles \\
        \textit{vehicules-immatricules-baac-YYYY.csv} & not used in this report \\
        \bottomrule
    \end{tabularx}
\end{table}
The French government publicly releases a dataset\footnote{\url{https://data.gouv.fr/en/datasets/bases-de-donnees-annuelles-des-accidents-corporels-de-la-circulation-routiere-annees-de-2005-a-2019/}} of  metadata on road accidents that occurred between~2005 and~2020. This dataset is released yearly and consist of 5~files per year in CSV format~(cf. Table~\ref{table-files}). Excluding one of the files\footnote{The \textit{vehicules-immatricules-baac-YYYY.csv} files are only available from 2010 to 2020.}, we download a total of 64~files for the 16~years considered for our visualizations. As shown in  Table~\ref{table-files}, the dataset feature a variety of dimensions. Many categorical dimensions describe characteristics of the accidents, vehicles, and persons involved. Using the categories we can filter or partition the dataset across multiple dimensions, in order to focus on specific characteristics in the dataset. The involved persons are also characterized regarding injuries and deadliness of the accident, which allows us to measure not only the number of accidents but also how severe each accident was. Similarly, each vehicle and each person involved in the accident is listed, so that we can count how much impact the accident likely had on the society. Crucially for measuring trends across years or months, each accident in the dataset is annotated with a timestamp. We therefore argue that the provided dataset is well suited to answer questions mentioned in Section~\ref{introduction}. Unfortunately, accidents, locations, vehicles, and persons are distributed in separate files. Hence, in our data preprocessing pipeline, we associate locations, vehicles, and persons to their matching accident.

After first attempts to parse and merge the CSV files directly inside the Elm application we decided to implement the data preprocessing in Python instead, as the type safety in Elm makes it complicated to correct errors in the government open data files. For example, CSV files from some years use different separators and even different file encodings. Also, some fields are named differently or use different number formats. In Python such errors are easier to debug and work around. By preprocessing the CSV files we also group persons by their vehicle and vehicles by the accident.Our Python preprocessing is structured in three parsers, for accidents, vehicles, and persons. After parsing, the three lists are grouped and exported. A smaller sample is extracted due to memory constraints when loading the dataset in the web browser.

\paragraph{Parsers}
\begin{listing}
    \lstinputlisting[linerange={17-17,32-36,43-45,48-48,59-59,74-77,110-111},language=Python]{../preprocessing/parse/person.py}
    \caption{Parser module for the CSV file containing person information. Function and constructor abbreviated.}
    \label{listing-data-parser}
\end{listing}
For each of the four data types, characteristics, locations, vehicles, and persons, we implement a parser module as illustrated in Listing~\ref{listing-data-parser}. The parser modules are responsible for opening the corresponding dataset file for the specified year and then reading the CSV records with the correct file encoding and separator. From the CSV records we then extract Python named tuples to improve type-safety for the JSON export. All columns are stripped from surrounding whitespace or illegal special characters~(e.g., the non-breaking space). For categorical columns~(cf. Listing~\ref{listing-data-parser}), we filter out invalid values and then map the values to Python \lstinline[language=Python]{Enum} classes containing the value coding from the French government's dataset description. Floating point values are parsed after correcting the decimal separator. We also split and parse lists of values contained in the CSV columns~(e.g., for the \lstinline[language=]{secu} column specifying safety equipment). For some columns, we also remove prefixes and suffixes~(e.g., for the upstream terminal~\lstinline[language=]{pr1}). The upstream terminal data is particularly difficult to parse because for some years the \lstinline[language=]{pr}~and \lstinline[language=]{pr1}~columns are swapped and need to be corrected. Each parser returns a list of instances of its corresponding named tuple together with the accident ID, vehicle ID, and person number if applicable.

\paragraph{Accident Grouping}
After parsing records of accidents, vehicles, and persons we match the vehicle and person records with the accidents. This is done by grouping persons by their vehicle ID and then associating each group of persons with the corresponding vehicle. Similarly, vehicles~(together with their associated persons) are then grouped by accident ID and then associating the vehicle groups with the matching accident record. This mapping from three different lists of records into a single list of records, containing lists of lists, makes parsing a single accidents easier in the Elm application.

\paragraph{Export and Sampling}
\label{export}
After preprocessing the data, we bundle the accident dataset as static resource, writing one JSON record per line to the exported JSONL file. For the conversion from Python's named tuple types that we modelled for each of the record types~(i.e., accident, vehicle, and person) to JSON, we convert the named tuples to Python \lstinline[language=Python]{Dict} instances which then can be converted to a JSON string using the built in \lstinline[language=Python]{json.dumps} function. Because of the large size of the aggregated dataset~(1065053~accidents/lines or 2~GB disk size), we also derive a smaller subset of the dataset by uniformly sampling 10\,000 random accidents from the whole dataset~(19~MB disk space). This way we extract a dataset that can be used with regular web browsers on common consumer hardware. To account for sampling biases, we test the visualizations with 3~different samples and find no major visual differences across the samples. We therefore assume that the sample dataset is able to sufficiently represent the whole dataset.

\section{Visualizations}
We propose three interactive visualizations by first analysing a set of tasks our application should solve~(cf. Section~\ref{tasks}), and then deriving specific requirements for the included visualizations~(cf. Section~\ref{requirements}). Furhtermore, we discus expressiveness, effectiveness, and appropriateness for each of the three implemented visualizations~(cf. Section~\ref{presentation}) and explain the interactivity of the visualization application~(cf. Section~\ref{interaction}).

\subsection{Analysis of Application Tasks}
\label{tasks}
\begin{table}
    \caption{Exemplary questions that should be answered by visualizions of French road accident dataset, with the typical background a person asking that question would likely have.}
    \label{table-questions}
    \begin{tabularx}{\linewidth}{cXp{0.3\linewidth}}
        \toprule
        \textbf{\#} & \textbf{Question} & \textbf{Background} \\
        \midrule
        1 & Are roads more dangerous in winter or summer? & citizen \\
        2 & Do older or younger people drive more safely? & policy maker \\
        3 & Are there hotspots of accidents in cities or rural areas? & citizen, policy maker \\
        4 & If hotspots occur, which other characteristics can explain the correlations. & policy maker \\
        5 & Do the proportion of dead and injured persons correlate? & policy maker \\
        6 & Where are unproportionately more people killed in traffic? & policy maker \\
        7 & Do dedicated bicycle lanes make roads safer for cyclists? & citizen, policy maker, infrastructure planner \\
        8 & Are wider roads safer than narrow roads? & citizen, infrastructure planner \\
        \bottomrule
    \end{tabularx}
\end{table}
Because of the diverse application background and target groups described in Section~\ref{introduction}, many tasks can be formulated to be answered by data visualizations. To narrow down the tasks, in this report, we propose a set of exemplary questions our visualizations should answer. The questions listed in Table~\ref{table-questions} also specify the most likely background a person asking that question would have. This additional context allows us to fine-tune individual visualizations by assuming background knowledge from the respective target group.
We identify 3~mental models as most important to help people answer the abovementioned questions: \Ni times and dates, \Nii geographical position, \Niii clustering or categorization.
Time and date, for example, are important to identify periodical trends, as required in question~1 in Table~\ref{table-questions}. A time series plot is able to capture the continuity of time and express a single data dimension in relation to time, which is enough for identifying the dangerousness of specific time periods.
Geographical position is crucial for understanding local patterns, as required in questions~3 or~6. While most people are very familiar with using maps as visualization for geographically positioned data, it is often not obvious how to visualize the multiple dimensions of the data points that are displayed on that map. We argue that given the space constraints of a map, the stick figures technique~\cite{PickettG1988} is appropriate because it can display a fixed number of dimensions in small space while observers can still detect patterns across the maps geographical dimensions.
And clustering accident occasions with respect to different categories is needed for comparing how different characteristics might influence road safety, as required in question~7. When working with multidimensional data, like we do with accidents, trees and treemaps~\cite{Shneiderman1992} can be a good visualization to present different, hierarchical categories.

\subsection{Visualization Requirements}
\label{requirements}
We derive 3~main goals that our visualization application should implement: \Ni to identify dangerous regions, times, and situations, \Nii to summarize trends with respect to time and location, and \Niii to offer detailed views where data is aggregated.
Identifying dangerous patterns should be supported by emphasizing road injuries or road deaths at first glance or by showing differences of accidents' characteristics with respect to visual reference points.
Accidents that follow a specific pattern should be distinguishable from other accidents.
Time-based trends should be made visible both in the long term and in the short term. Geographical trends must also include the geographical position but should not over-emphasize it to be effective.
And to avoid losing important details such as outliers, the visualizations should always feature a way to show the data with as little aggregation~(e.g., average, minimum, maximum) as possible, while still not overwhelming observers of the visualization.

\subsection{Visualization Presentation}
\label{presentation}
The visualization application proposed to solve the aforementioned tasks~(cf. Section~\ref{tasks}) consists of three interatively combined visualization pages: \Ni a time series plot to compare the number of accidents or casualties, \Nii a geographical map of icon-based stick figure visualizations, and \Niii a treemap view of hierarchically categorized accidents.

\subsubsection{First Visualization}
\todo{Präsentieren sie die visuellen Abbildungen und Kodierungen der Daten und Interaktionsmöglichkeiten.
Sie müssen begründen, warum und wie gut ihre Designentscheidungen die erstellten Anforderungen erfüllen.
Weiterhin müssen sie begründen, warum die gewählte visuelle Kodierung der Daten für das zu lösenden Problem passend ist.
Typische Argumente würden hier auf Wahrnehmungsprinzipien und Theorie über Informationsvisualisierung verweisen. 
Die besten Begründungen diskutieren explizit die konkrete Auswahl der Visualisierungen im Kontext von mehreren verschiedenen Alternativen. 
Machen Sie hier nicht den Fehler, einfach nur Visualisierung aus den vorgegebenen Bereichen zu diskutieren, weil das in der Regel nicht sinnvoll ist.
Wenn sie sich für einen Scatterplot entschieden haben, ist ein Zeitreihendiagramm in der Regel keine Alternative.
Diskutieren Sie also nicht einfach Zeitreihendiagramme, weil sie in den Anforderungen an das Projekt neben Scatterplots stehen, sondern suchen sie nach echten alternativen Visualisierungen, die zum Aufbau eines vergleichbaren mentalen Modells führen.
Diskutieren Sie die Expressivität und die Effektivität der einzelnen Visualisierungen.}

\subsubsection{Second Visualization}
We use a equirectanguular map projection of france\footnote{\url{https://commons.wikimedia.org/wiki/File:France_location_map-Regions_and_departements-2016.svg}} as the background to the second visualization.
\todo{Präsentieren sie die visuellen Abbildungen und Kodierungen der Daten und Interaktionsmöglichkeiten.
Sie müssen begründen, warum und wie gut ihre Designentscheidungen die erstellten Anforderungen erfüllen.
Weiterhin müssen sie begründen, warum die gewählte visuelle Kodierung der Daten für das zu lösenden Problem passend ist.
Typische Argumente würden hier auf Wahrnehmungsprinzipien und Theorie über Informationsvisualisierung verweisen. 
Die besten Begründungen diskutieren explizit die konkrete Auswahl der Visualisierungen im Kontext von mehreren verschiedenen Alternativen. 
Machen Sie hier nicht den Fehler, einfach nur Visualisierung aus den vorgegebenen Bereichen zu diskutieren, weil das in der Regel nicht sinnvoll ist.
Wenn sie sich für einen Scatterplot entschieden haben, ist ein Zeitreihendiagramm in der Regel keine Alternative.
Diskutieren Sie also nicht einfach Zeitreihendiagramme, weil sie in den Anforderungen an das Projekt neben Scatterplots stehen, sondern suchen sie nach echten alternativen Visualisierungen, die zum Aufbau eines vergleichbaren mentalen Modells führen.
Diskutieren Sie die Expressivität und die Effektivität der einzelnen Visualisierungen.}

\subsubsection{Third Visualization}
\todo{Präsentieren sie die visuellen Abbildungen und Kodierungen der Daten und Interaktionsmöglichkeiten.
Sie müssen begründen, warum und wie gut ihre Designentscheidungen die erstellten Anforderungen erfüllen.
Weiterhin müssen sie begründen, warum die gewählte visuelle Kodierung der Daten für das zu lösenden Problem passend ist.
Typische Argumente würden hier auf Wahrnehmungsprinzipien und Theorie über Informationsvisualisierung verweisen. 
Die besten Begründungen diskutieren explizit die konkrete Auswahl der Visualisierungen im Kontext von mehreren verschiedenen Alternativen. 
Machen Sie hier nicht den Fehler, einfach nur Visualisierung aus den vorgegebenen Bereichen zu diskutieren, weil das in der Regel nicht sinnvoll ist.
Wenn sie sich für einen Scatterplot entschieden haben, ist ein Zeitreihendiagramm in der Regel keine Alternative.
Diskutieren Sie also nicht einfach Zeitreihendiagramme, weil sie in den Anforderungen an das Projekt neben Scatterplots stehen, sondern suchen sie nach echten alternativen Visualisierungen, die zum Aufbau eines vergleichbaren mentalen Modells führen.
Diskutieren Sie die Expressivität und die Effektivität der einzelnen Visualisierungen.}

\subsection{Interaction}
\label{interaction}
\todo{Die präsentierten Visualisierungstechniken müssen interaktiv zu einer Anwendung verknüpft werden.
Die Interaktion mit einer Visualisierung soll in den anderen Visualisierungen zu einer Änderung führen. 
Erklären Sie die möglichen Interaktionen mit den einzelnen Visualisierungen und die möglichen Verknüpfungen zwischen ihnen. Begründen Sie, warum die konkreten Interaktionen umgesetzt wurden und welche Zwecke für die Anwenderinnen mit ihnen unterstützt werden. Begründen Sie ebenfalls, warum sie andere Interaktionsmöglichkeiten nicht umgesetzt haben. Wenn sie keine der geforderten Interaktionen umsetzen, erhalten Sie im gesamten Projekt deutlichen Punktabzug.}

\section{Implementation}
\todo{Beschreiben Sie die Implementierung ihrer Visualisierungs-Anwendung in Elm. Stellen die Gliederung ihres Quellcodes vor. Haben Sie verschiedene Elm-Module erstellt. Was war aufwändig umzusetzen, was ließ sich mit dem vorhanden Code aus den Übungen relativ einfach umsetzen?}

\todo{Wie sieht die Elm-Datenstruktur für das Model aus, in dem die verschiedenen Zustände der Interaktion gespeichert werden können.}


\section{Use Cases}
\todo{Präsentieren sie für jede der drei Visualisierungen einen sinnvollen Anwendungsfall in dem ein bestimmter Fakt, ein Muster oder die Abwesenheit eines Musters visuell festgestellt wird. Begründen Sie, warum dieser Anwendungsfall wichtig für die Zielgruppe der Anwenderinnen ist. Diskutieren Sie weiterhin, ob die oben beschriebene Information auch mit anderen Visualisierungstechniken hätte gefunden werden können. Falls dies möglich wäre, vergleichen sie die den Aufwand und die Schwierigkeiten ihres Ansatzes und möglicher Alternativen.}
\subsection{First Visualization Use Case}
\subsection{Second Visualization Use Case}
\subsection{Third Visualization Use Case}

\section{Related Work}
\todo{Führen sie eine kurze Literatursuche in der wissenschaftlichen Literatur zu Informationsvisualisierung und Visual Analytics nach ähnlichen Anwendungen durch. Diskutieren Sie mindestens zwei Artikel. Stellen Sie Gemeinsamkeiten und Unterschiede dar.}

\section{Conclusion and Future Work}
\todo{Fassen sie die Beiträge ihre Visualisierungs-Anwendung zusammen. Wo bietet sie für die Personen der Zielgruppe einen echten Mehrwert?}

\todo{Was wären mögliche sinnvolle Erweiterungen, entweder auf der Ebene der Visualisierungen und/oder auf der Datenebene?}

\section*{Appendix: Git History}
% Generate with: git --no-pager log --all --graph --no-color --date=short --pretty='format:%h%d (%s, %ad)'
% \lstinputlisting[language=,basicstyle=\ttfamily\tiny]{git-log.txt}

\printbibliography

\end{document}

