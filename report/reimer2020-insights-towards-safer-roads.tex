\documentclass[usegeometry=true]{scrartcl}
\usepackage[english]{babel}
\usepackage[T1]{fontenc}
\usepackage{lmodern}
\usepackage[utf8]{inputenc}
\usepackage{hyperref}
\usepackage{amssymb}
% Don't change dimensions.
\usepackage[left=2cm, right=2cm, top=2cm, bottom=2cm, bindingoffset=1cm, includeheadfoot]{geometry}
% Don't change spacing.
\usepackage[onehalfspacing]{setspace}
\usepackage[backend=biber,style=numeric]{biblatex}
\usepackage{csquotes}
\addbibresource{literature.bib}
\begin{document}

\subject{Project Report for the Module \\ \textquote{Information Retrieval und Visualisierung} \\ in Summer Semester~2022}
\title{Insights Towards Safer Roads}
\subtitle{Visualizing Accidents in France from~2005 to~2020}
\author{Jan Heinrich Reimer}
\date{\today}
\maketitle

%\tableofcontents

\section{Introduction}
Der Bericht fällt in die Kategorie von InfoVis-Paper, die \textcite{Munzner2008} Design Study nennt: In der Einleitung sollen sie zuerst das Zielproblem beschrieben. Daraus sollen sie Fragestellungen motivieren, die mittels Techniken der Informationsvisualisierung beantwortet werden können. In dem Abschnitt direkt unter der Überschrift Einleitung sollen Sie nach einer kurzen Einleitung Fragestellungen und das Zielproblem motivieren und beschreiben.

\subsection{Application Background}
Sie müssen genug Hintergrund bereitstellen, sodass die Lesenden sich ein Urteil bilden können, ob ihre Lösung funktioniert. Sie sollen die Lesenden jedoch nicht mit Anwendungsdetails so überschütten, dass der Fokus auf die Fragen zur Informationsvisualisierung untergehen.
\subsection{Target Groups}
Beschreiben sie die Personengruppe oder Personengruppen, die das von ihnen benannte Anwendungsproblem lösen möchte. Auf welches Vorwissen können sie in dieser Gruppen von Anwenderinnen aufbauen? Welche Informationsbedürfnisse werden durch die Visualisierungen adressiert?
\subsection{Overview and Contributions}
In diesem Abschnitt geben sie einen kurzen Überblick über die Daten und verwendeten Visualisierungen. Dann benennen sie die Beiträge ihres Projekts. Diese Beiträge müssen sie in den hinteren Teilen des Berichts genauer ausführen und belegen.

\section{Data}
Beschreiben Sie vorhandenen Daten. Gehen Sie kritisch darauf ein, inwieweit sich die Daten für die Bearbeitung der Fragestellungen und dem Erreichen von Lösungen für die oben beschriebene Zielgruppen eignen. Haben sie die Daten sinnvoll mit weiteren Datenquellen ergänzt? Wenn ja, wie?
Erklären Sie die technische Bereitstellung der Daten.
Wie sind die Daten zugänglich? Welche Formate werden genutzt. Gibt es Besonderheiten beim Lesen der Formate?
Beschreiben Sie die Datenvorverarbeitung.
Welche Datenvorverarbeitungsschritte sind notwendig? Beschreiben Sie die einzelnen Schritte und begründen Sie sie, zum Beispiel warum werden manche Daten weggelassen, über welche Mengen werden Durchschnitte berechnet, warum sind die so berechneten Werte aussagekräftiger als andere Werte. Wenn möglich sollen sie die Datenvorverarbeitung in Elm programmieren, sodass ihre Anwendung auf eine Änderung der Rohdaten reagieren kan.

\section{Visualizations}
\subsection{Analysis of Application Tasks}
Analysieren sie die konkreten Anwendungsaufgaben, die die Lösung des Zielproblems durch die Anwender:innen bearbeitet werden müssen. 
Welche sinnvollen mentalen Modelle helfen den Personen bei der Bearbeitung.
%Welche Visualisierungen helfen den Personen, die die Software verwenden, sinnvolle mentale Modelle aufzubauen. 
Sind diese mentalen Modelle für sie notwendig, um die Aufgaben lösen zu können? Gehen Sie bei ihrer Argumentation von den Anwendungsaufgaben aus und kommen sie dann zu den mentalen Modellen, deren Aufbau durch Visualisierungen unterstützt wird.
\subsection{Visualization Requirements}
Leiten sie Anforderungen an das Design der Visualisierungen ab, die sich durch ihre Analyse des Zielproblems ergeben.
\subsection{Visualization Presentation}
Präsentieren sie die visuellen Abbildungen und Kodierungen der Daten und Interaktionsmöglichkeiten.
Sie müssen begründen, warum und wie gut ihre Designentscheidungen die erstellten Anforderungen erfüllen.
Weiterhin müssen sie begründen, warum die gewählte visuelle Kodierung der Daten für das zu lösenden Problem passend ist.
Typische Argumente würden hier auf Wahrnehmungsprinzipien und Theorie über Informationsvisualisierung verweisen. 
Die besten Begründungen diskutieren explizit die konkrete Auswahl der Visualisierungen im Kontext von mehreren verschiedenen Alternativen. 
Machen Sie hier nicht den Fehler, einfach nur Visualisierung aus den vorgegebenen Bereichen zu diskutieren, weil das in der Regel nicht sinnvoll ist.
Wenn sie sich für einen Scatterplot entschieden haben, ist ein Zeitreihendiagramm in der Regel keine Alternative.
Diskutieren Sie also nicht einfach Zeitreihendiagramme, weil sie in den Anforderungen an das Projekt neben Scatterplots stehen, sondern suchen sie nach echten alternativen Visualisierungen, die zum Aufbau eines vergleichbaren mentalen Modells führen.
Diskutieren Sie die Expressivität und die Effektivität der einzelnen Visualisierungen.

Die eben beschriebenen Präsentationen und Begründungen sollen für jede der drei folgenden Visualisierungen durchgeführt werden. 
\subsubsection{First Visualization}
\subsubsection{Second Visualization}
\subsubsection{Third Visualization}

\subsection{Interaction}
Die präsentierten Visualisierungstechniken müssen interaktiv zu einer Anwendung verknüpft werden.
Die Interaktion mit einer Visualisierung soll in den anderen Visualisierungen zu einer Änderung führen. 
Erklären Sie die möglichen Interaktionen mit den einzelnen Visualisierungen und die möglichen Verknüpfungen zwischen ihnen. Begründen Sie, warum die konkreten Interaktionen umgesetzt wurden und welche Zwecke für die Anwenderinnen mit ihnen unterstützt werden. Begründen Sie ebenfalls, warum sie andere Interaktionsmöglichkeiten nicht umgesetzt haben. Wenn sie keine der geforderten Interaktionen umsetzen, erhalten Sie im gesamten Projekt deutlichen Punktabzug.

\section{Implementation}
Beschreiben Sie die Implementierung ihrer Visualisierungs-Anwendung in Elm. Stellen die Gliederung ihres Quellcodes vor. Haben Sie verschiedene Elm-Module erstellt. Was war aufwändig umzusetzen, was ließ sich mit dem vorhanden Code aus den Übungen relativ einfach umsetzen?

Wie sieht die Elm-Datenstruktur für das Model aus, in dem die verschiedenen Zustände der Interaktion gespeichert werden können.

\section{Use Cases}
Präsentieren sie für jede der drei Visualisierungen einen sinnvollen Anwendungsfall in dem ein bestimmter Fakt, ein Muster oder die Abwesenheit eines Musters visuell festgestellt wird. Begründen Sie, warum dieser Anwendungsfall wichtig für die Zielgruppe der Anwenderinnen ist. Diskutieren Sie weiterhin, ob die oben beschriebene Information auch mit anderen Visualisierungstechniken hätte gefunden werden können. Falls dies möglich wäre, vergleichen sie die den Aufwand und die Schwierigkeiten ihres Ansatzes und möglicher Alternativen.
\subsection{First Visualization Use Case}
\subsection{Second Visualization Use Case}
\subsection{Third Visualization Use Case}

\section{Related Work}
Führen sie eine kurze Literatursuche in der wissenschaftlichen Literatur zu Informationsvisualisierung und Visual Analytics nach ähnlichen Anwendungen durch. Diskutieren Sie mindestens zwei Artikel. Stellen Sie Gemeinsamkeiten und Unterschiede dar.

\section{Conclusion and Future Work}
Fassen sie die Beiträge ihre Visualisierungs-Anwendung zusammen. Wo bietet sie für die Personen der Zielgruppe einen echten Mehrwert?

Was wären mögliche sinnvolle Erweiterungen, entweder auf der Ebene der Visualisierungen und/oder auf der Datenebene?

\section*{Appendix: Git History}

\printbibliography

\end{document}

