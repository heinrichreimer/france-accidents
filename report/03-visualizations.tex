\section{Visualizations}
\subsection{Analysis of Application Tasks}
\todo{Analysieren sie die konkreten Anwendungsaufgaben, die die Lösung des Zielproblems durch die Anwender:innen bearbeitet werden müssen. 
Welche sinnvollen mentalen Modelle helfen den Personen bei der Bearbeitung.
Welche Visualisierungen helfen den Personen, die die Software verwenden, sinnvolle mentale Modelle aufzubauen. 
Sind diese mentalen Modelle für sie notwendig, um die Aufgaben lösen zu können? Gehen Sie bei ihrer Argumentation von den Anwendungsaufgaben aus und kommen sie dann zu den mentalen Modellen, deren Aufbau durch Visualisierungen unterstützt wird.}
\subsection{Visualization Requirements}
\todo{Leiten sie Anforderungen an das Design der Visualisierungen ab, die sich durch ihre Analyse des Zielproblems ergeben.}
\subsection{Visualization Presentation}
\todo{Präsentieren sie die visuellen Abbildungen und Kodierungen der Daten und Interaktionsmöglichkeiten.
Sie müssen begründen, warum und wie gut ihre Designentscheidungen die erstellten Anforderungen erfüllen.
Weiterhin müssen sie begründen, warum die gewählte visuelle Kodierung der Daten für das zu lösenden Problem passend ist.
Typische Argumente würden hier auf Wahrnehmungsprinzipien und Theorie über Informationsvisualisierung verweisen. 
Die besten Begründungen diskutieren explizit die konkrete Auswahl der Visualisierungen im Kontext von mehreren verschiedenen Alternativen. 
Machen Sie hier nicht den Fehler, einfach nur Visualisierung aus den vorgegebenen Bereichen zu diskutieren, weil das in der Regel nicht sinnvoll ist.
Wenn sie sich für einen Scatterplot entschieden haben, ist ein Zeitreihendiagramm in der Regel keine Alternative.
Diskutieren Sie also nicht einfach Zeitreihendiagramme, weil sie in den Anforderungen an das Projekt neben Scatterplots stehen, sondern suchen sie nach echten alternativen Visualisierungen, die zum Aufbau eines vergleichbaren mentalen Modells führen.
Diskutieren Sie die Expressivität und die Effektivität der einzelnen Visualisierungen.}

\todo{Die eben beschriebenen Präsentationen und Begründungen sollen für jede der drei folgenden Visualisierungen durchgeführt werden. }
\subsubsection{First Visualization}
\subsubsection{Second Visualization}
We used a freely licensed equirectanguular map projection of france\footnote{\url{https://commons.wikimedia.org/wiki/File:France_location_map-Regions_and_departements-2016.svg}} as the background to the second visualization.
\subsubsection{Third Visualization}

\subsection{Interaction}
\todo{Die präsentierten Visualisierungstechniken müssen interaktiv zu einer Anwendung verknüpft werden.
Die Interaktion mit einer Visualisierung soll in den anderen Visualisierungen zu einer Änderung führen. 
Erklären Sie die möglichen Interaktionen mit den einzelnen Visualisierungen und die möglichen Verknüpfungen zwischen ihnen. Begründen Sie, warum die konkreten Interaktionen umgesetzt wurden und welche Zwecke für die Anwenderinnen mit ihnen unterstützt werden. Begründen Sie ebenfalls, warum sie andere Interaktionsmöglichkeiten nicht umgesetzt haben. Wenn sie keine der geforderten Interaktionen umsetzen, erhalten Sie im gesamten Projekt deutlichen Punktabzug.}
