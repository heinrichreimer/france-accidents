\section{Visualizations}
We propose three interactive visualizations by first analysing a set of tasks our application should solve~(cf. Section~\ref{tasks}), and then deriving specific requirements for the included visualizations~(cf. Section~\ref{requirements}). Furhtermore, we discus expressiveness, effectiveness, and appropriateness for each of the three implemented visualizations~(cf. Section~\ref{presentation}) and explain the interactivity of the visualization application~(cf. Section~\ref{interaction}).

\subsection{Analysis of Application Tasks}
\label{tasks}
\begin{table}
    \caption{Exemplary questions that should be answered by visualizions of French road accident dataset, with the typical background a person asking that question would likely have.}
    \label{table-questions}
    \begin{tabularx}{\linewidth}{cXp{0.3\linewidth}}
        \toprule
        \textbf{\#} & \textbf{Question} & \textbf{Background} \\
        \midrule
        1 & Are roads more dangerous in winter or summer? & citizen \\
        2 & Do older or younger people drive more safely? & policy maker \\
        3 & Are there hotspots of accidents in cities or rural areas? & citizen, policy maker \\
        4 & If hotspots occur, which other characteristics can explain the correlations. & policy maker \\
        5 & Do the proportion of dead and injured persons correlate? & policy maker \\
        6 & Where are unproportionately more people killed in traffic? & policy maker \\
        7 & Do dedicated bicycle lanes make roads safer for cyclists? & citizen, policy maker, infrastructure planner \\
        8 & Are wider roads safer than narrow roads? & citizen, infrastructure planner \\
        \bottomrule
    \end{tabularx}
\end{table}
Because of the diverse application background and target groups described in Section~\ref{introduction}, many tasks can be formulated to be answered by data visualizations. To narrow down the tasks, in this report, we propose a set of exemplary questions our visualizations should answer. The questions listed in Table~\ref{table-questions} also specify the most likely background a person asking that question would have. This additional context allows us to fine-tune individual visualizations by assuming background knowledge from the respective target group.
We identify 3~mental models as most important to help people answer the abovementioned questions: \Ni times and dates, \Nii geographical position, \Niii clustering or categorization.
Time and date, for example, are important to identify periodical trends, as required in question~1 in Table~\ref{table-questions}. A time series plot is able to capture the continuity of time and express a single data dimension in relation to time, which is enough for identifying the dangerousness of specific time periods.
Geographical position is crucial for understanding local patterns, as required in questions~3 or~6. While most people are very familiar with using maps as visualization for geographically positioned data, it is often not obvious how to visualize the multiple dimensions of the data points that are displayed on that map. We argue that given the space constraints of a map, the stick figures technique~\cite{PickettG1988} is appropriate because it can display a fixed number of dimensions in small space while observers can still detect patterns across the maps geographical dimensions.
And clustering accident occasions with respect to different categories is needed for comparing how different characteristics might influence road safety, as required in question~7. When working with multidimensional data, like we do with accidents, trees and treemaps~\cite{Shneiderman1992} can be a good visualization to present different, hierarchical categories.

\subsection{Visualization Requirements}
\label{requirements}
We derive 3~main goals that our visualization application should implement: \Ni to identify dangerous regions, times, and situations, \Nii to summarize trends with respect to time and location, and \Niii to offer detailed views where data is aggregated.
Identifying dangerous patterns should be supported by emphasizing road injuries or road deaths at first glance or by showing differences of accidents' characteristics with respect to visual reference points.
Accidents that follow a specific pattern should be distinguishable from other accidents.
Time-based trends should be made visible both in the long term and in the short term. Geographical trends must also include the geographical position but should not over-emphasize it to be effective.
And to avoid losing important details such as outliers, the visualizations should always feature a way to show the data with as little aggregation~(e.g., average, minimum, maximum) as possible, while still not overwhelming observers of the visualization.

\subsection{Visualization Presentation}
\label{presentation}
\todo{Präsentieren sie die visuellen Abbildungen und Kodierungen der Daten und Interaktionsmöglichkeiten.
Sie müssen begründen, warum und wie gut ihre Designentscheidungen die erstellten Anforderungen erfüllen.
Weiterhin müssen sie begründen, warum die gewählte visuelle Kodierung der Daten für das zu lösenden Problem passend ist.
Typische Argumente würden hier auf Wahrnehmungsprinzipien und Theorie über Informationsvisualisierung verweisen. 
Die besten Begründungen diskutieren explizit die konkrete Auswahl der Visualisierungen im Kontext von mehreren verschiedenen Alternativen. 
Machen Sie hier nicht den Fehler, einfach nur Visualisierung aus den vorgegebenen Bereichen zu diskutieren, weil das in der Regel nicht sinnvoll ist.
Wenn sie sich für einen Scatterplot entschieden haben, ist ein Zeitreihendiagramm in der Regel keine Alternative.
Diskutieren Sie also nicht einfach Zeitreihendiagramme, weil sie in den Anforderungen an das Projekt neben Scatterplots stehen, sondern suchen sie nach echten alternativen Visualisierungen, die zum Aufbau eines vergleichbaren mentalen Modells führen.
Diskutieren Sie die Expressivität und die Effektivität der einzelnen Visualisierungen.}

\todo{Die eben beschriebenen Präsentationen und Begründungen sollen für jede der drei folgenden Visualisierungen durchgeführt werden. }

\subsubsection{First Visualization}

\subsubsection{Second Visualization}
We use a equirectanguular map projection of france\footnote{\url{https://commons.wikimedia.org/wiki/File:France_location_map-Regions_and_departements-2016.svg}} as the background to the second visualization.

\subsubsection{Third Visualization}

\subsection{Interaction}
\label{interaction}
\todo{Die präsentierten Visualisierungstechniken müssen interaktiv zu einer Anwendung verknüpft werden.
Die Interaktion mit einer Visualisierung soll in den anderen Visualisierungen zu einer Änderung führen. 
Erklären Sie die möglichen Interaktionen mit den einzelnen Visualisierungen und die möglichen Verknüpfungen zwischen ihnen. Begründen Sie, warum die konkreten Interaktionen umgesetzt wurden und welche Zwecke für die Anwenderinnen mit ihnen unterstützt werden. Begründen Sie ebenfalls, warum sie andere Interaktionsmöglichkeiten nicht umgesetzt haben. Wenn sie keine der geforderten Interaktionen umsetzen, erhalten Sie im gesamten Projekt deutlichen Punktabzug.}
