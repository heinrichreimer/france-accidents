\section{Implementation}
\label{implementation}
\begin{listing}
    \lstinputlisting[linerange={76-78,98-103}]{../src/Main.elm}
    \caption{Example of wrapping the \lstinline{update} function from \lstinline{Visualization1} inside the \lstinline{Main} module's \lstinline{update} function. Commands sent from \lstinline{Visualization1} are then mapped to the \lstinline{Main} modules types.}
    \label{listing-map-model}
\end{listing}
Our demo application is developed in the Elm programming language\footnote{\url{https://elm-lang.org/}} following the functional proramming paradigm. We implement each visualization as a sub-application in a separate Elm module~(\lstinline{Visualization1}, \lstinline{Visualization2}, \lstinline{Visualization3}) that only exposes the functions required in the Elm application architecture~(i.e., \lstinline{Model}, \lstinline{Msg}, \lstinline{init}, \lstinline{update}, \lstinline{view}) as well as a visualization name~(i.e., \lstinline{label}). The \lstinline{Main} module is then responsible for switching visualizations and any bundling messages sent from either of the sub-applications, by mapping and wrapping each visuzlization's model, messages, and view as illustrated in Listing~\ref{listing-map-model}. In addition to the modules used for displaying the application, additional modules are implemented for defining the accident model~(\lstinline{Model}), parsing accidents from the JSONL~files described in Section~\ref{data}~(\lstinline{Data}), and for utility functions and data structures~(\lstinline{Utils}, \lstinline{TimeUtils}, \lstinline{TreeUtils}, \lstinline{Partition}).

\paragraph{Model}

\paragraph{Data}
We then load the static JSONL dataset file~(cf. Section~\ref{data}) via HTTP requests and parse the response line-by-line into a list of type-safe data structures. Parsers are implemented in the \lstinline{Json.Decode} framework by creating deoders for each categorical dimension, the \lstinline{Accident} structure, the \lstinline{Vehicle} structure, and the \lstinline{Person} stucture.
% Examples 

\paragraph{Utilities and Additional Data Structures}


\paragraph{First Visualization}

\paragraph{First Visualization2}

\paragraph{First Visualization}

\todo{Beschreiben Sie die Implementierung ihrer Visualisierungs-Anwendung in Elm. Stellen die Gliederung ihres Quellcodes vor. Haben Sie verschiedene Elm-Module erstellt. Was war aufwändig umzusetzen, was ließ sich mit dem vorhanden Code aus den Übungen relativ einfach umsetzen?}

\todo{Wie sieht die Elm-Datenstruktur für das Model aus, in dem die verschiedenen Zustände der Interaktion gespeichert werden können.}
